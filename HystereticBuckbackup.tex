\documentclass[conference]{IEEEtran}

\usepackage{cite}
\usepackage[pdftex]{graphicx}
\usepackage{amsmath}
\usepackage{array}

\usepackage{eqparbox}
\usepackage[tight,footnotesize]{subfigure}
\usepackage[font=footnotesize]{subfig}
\usepackage{caption}
\usepackage{fixltx2e}
\usepackage{stfloats}

\usepackage{url}


% correct bad hyphenation here
\hyphenation{op-tical net-works semi-conduc-tor}


\begin{document}

\title{Ultra-Simple, Two Switch, Self-Oscillating, Hysteretic Current Controlled Buck Converter, ideal for LEDs}

\author{%
\IEEEauthorblockN{Pete Scourboutakos, Steve Mann, Mir Adnan Ali, Ryan Janzen}
\IEEEauthorblockA{Dept.\ of Electrical and Computer Engineering\\
University of Toronto\\ Toronto, Canada\\
{\footnotesize\texttt{\{pete, mann, maali, ryan\}{@}eyetap.org}}}}
\maketitle


\begin{abstract}
With LEDs displacing other forms of lighting at an ever increasing pace, they deserve a ubiquitously simple and universal power supply to facilitate their usage wherever needed, and despite the form of the power supplied. Thus we present an ultra-simple, two switch, self-oscillating, hysteretic current controlled buck converter which provides a regulated current for a single LED or string of LEDs. Several experimental prototypes have been built and tested to operate with excellent output regulation and efficiency from AC or DC power from 10 to 340V input. 
\end{abstract}

% For peer review papers, you can put extra information on the cover
% page as needed:
% \ifCLASSOPTIONpeerreview
% \begin{center} \bfseries EDICS Category: 3-BBND \end{center}
% \fi
%
% For peerreview papers, this IEEEtran command inserts a page break and
% creates the second title. It will be ignored for other modes.
\IEEEpeerreviewmaketitle

\section{Introduction}

In these times we observe a global movement toward the adoption of LED lighting technology. Whereas installed large-scale tubular fluorescent lighting may remain in use for some time, the future is near where LED lighting will very soon displace incandescant and CFL lighting in most of their applications. LED technology is preferred due to its very high efficiency and long lifetime, and as such it will become the light source of choice in time, as costs per light output continue to drop in relation to the older technologies. (And especially as payback times become reasonable, based on upfront cost for fixture recouped by long term electricity savings.)

The incandescant lamp has been the lighting standard for decades, and prefers to be driven by a regulated voltage[XXX]. As such, incandescant lamps are readily and easily connected directly to the (voltage regulated) power grid or through a common (single switch design) triac phase cutting dimmer circuit[XXX], which easily adjusts the light output power without having a great deal of effect on the power factor or anything else (for that matter). 

Unfortunately, other (more efficient) lighting technologies are not as happy to be connected directly to our electricity grid, and this is because these technologies prefer to be driven by a regulated current, and not voltage as supplied by our electricity grid.

For instance, CFLs include an active, current regulating ballast circuit. This is a ubiquitous circuit which is found to be nearly identical in every case. It is commonly a two transistor (most commonly npn bjt), self oscillating, push pull, resonant inverter which provides a regulated output current, suitable for the excitation of the CF bulb.

LED technology also prefers to be driven by a regulated current, however no such ubiquitous driver circuit exists as of yet. Recently there have been a large number of publications in this area. Many designs are put forth which aim to lack electrolytic capacitors and have good power factor. However many of these designs suffer from issues such as a small accepted input voltage range \cite{tianfu2007improved}, they require transformers \cite{chen2012elimination}, they use discrete controller ICs \cite{gu2009means}, and have multiple active power stages \cite{wang2012flicker}. 

The ideal power supply for LEDs operates from a very wide range of supply voltages both AC and DC, lacks electrolytic capacitors (has long lifetime), has acceptable power factor and provides a regulated output current in a simple way.


\subsection{Design}
% XXX how do we map known components to create a design matching 
% our ideal properties?
In an endeavor to fulfill these needs, we present an ultra-simple buck converter. This design powers an LED or string of LEDs with a constant and adjustable current. It operates over a tested input voltage range of 10 to 380V, AC or DC, with no electrolytic capacitor, no transformer, no auxiliary power supply, oscillator or controller. [XXX what is the novel work]. It is a switching version of a classic 2-transistor linear constant current source, used in amplifiers since the 1960s[seen here, must find original source.. 
\footnote{\url{http://ieeexplore.ieee.org/stamp/stamp.jsp?tp=&arnumber=1018}}
]. This current source comprises two transistor switches and a shunt resistor. It operates on the simple principle that transistor Q2 measures the current through the shunt resistor and accordingly adjusts the control of transistor Q1 to regulate this, the output current, to a particular level, as set by the resistance of the shunt resistor and the threshold level of transistor Q2.

We have made a small addition to this circuit, by adding resistors R3 and R4. R4 provides a path for hysteretic positive feedback. Positive feedback gives the circuit the ability to oscillate, and by the very nature of the current regulating circuit that it is, it oscillates with a duty ratio which inherenty binds the output current within limits of a regulated value, these limits are set by R3 and R4 which set the hyteresis ratio. In other words, it operates as a hysteresis (bang-bang) controller. In words the operation of the hysteretic converter is such that, with the output controlling switch (Q1) beginning on, the output current rises. At some moment (time1) the output current reaches a threshold (Vthreshold) above the setpoint, and this causes the output controlling switch (Q1) to switch off. Subsequently, the output current falls, and at the moment (time2) the output current falls a threshold (-Vthreshold) below the setpoint, the output controlling switch (Q1) is switched back on again, and the process repeats. 

By adding an output filter (inductor and capacitor) and freewheeling diode, the circuit is transformed into a buck converter with a regulated and smoothed output current. This means that the circuit may be connected to power a series string of LEDs, irrespective of how many LEDs are in the string, so long as the total operating voltage is less than the supply voltage.

The output current regulating setpoint is a function of both the threshold voltage of Q2 and the resistance of R1. Thus, the output current may be adjusted by an adjustment of the resistance of R1. This could be implemented with selection switches which select different resistors, or with an adjustable rheostat in series with a maximum current limit setting fixed resistor. (this must be designed carefully, should we mention that?)

The advantage of the load (the LED) being a bottleneck for all current passing through the device, is in device efficiency. However, the disadvantage is that there is always current trickling through the device and it cannot be dimmed below a certain brightness. Thus, a good old fahioned on/off switch is recommended on this lamp, as it also saves on the idle phantom quescent current which has grown to be accepted from the devices we leave plugged in everyday in life. 



\subsection{Implementation}
% should be called construction? XXX how do we create a prototype that can scale to production? 
% e.g. use all MOSFET technology so can create an IC
The circuit was implemented, tested and simulated with both mosfets and bipolar junction transistors. Both have advantages and disadvantages, however an optimal design has Q1 as a mosfet and Q2 as a bjt. Q2 is preferred to be a bjt because the bjt threshold voltage of about 0.7V is much lower than the 3 to 5V threshold of mosfets. A smaller threshold voltage means that the voltage required across the shunt resistor is less for the same output current, and thus the shunt resistance is reduced, along with its power dissipation. This greatly increases the efficiency of the overall circuit.

For the example case presented, FQN1N60CTA mosfets were used. Zener diodes must be added across the mosfet gate/sources in order to clamp this voltage and prevent it from rising to the point where it breaks down and damages the gate oxide of the transistor. The maximum output current was set to 150mA, an industry standard current for LEDs(truth?). This is accomplished by selecting the shunt resistor to be 27ohms, since the mosfets used had a threshold voltage of about 4V, 4V / 27ohms = 150mA. A 100ohm potentiometer, wired as a rheostat in series with the 27ohm resistor allows the current to be adjusted down to a small fraction of the maximum, dimming the light. 

The converter implemented with mosfets has been tested to operate over a very wide range of input supply voltages, and with the addition of a simple bridge rectifier, it will operate on AC or DC (of either polarity), potentially from 5 to over 1000V. This means that with this power converter, the same LED will run with the same light output from either a 12V car battery or a 347VAC light power source, or anything in between. [XXX PETE TEST AT 347 AND UP!]

This design works fine with no output capacitor (making the entire circuit lack capacitors completely), as long as the supply power is constant. With inconsistent power supplies such as single phase AC power, the capacitor supplies the energy needed to keep the LED lit during the portions of time where the power drops to zero. Luckily, these portions of time are so small that the capacitance needed is not of a great amount, and thus ceramic capacitors can be used instead of electrolytic.

Experiments were done with no added inductor or capacitor, simply utilizing the self inductance of the LED itself as the output filter. The circuit was found to work properly in this state, however oscillating at about 12.5Mhz, the switching losses were much increased and the efficiency was thus not favourable. 

The switching frequency is a function of all parts of the circuit, but a strong function of the amount of inductance present, and therefore an inductor should be chosen such that the switching frequency is reasonable. With a 100uH inductor recovered from a CFL recovered from the garbage, the switching frequency was about 150kHz, which allows for acceptable losses at about 150mA output. 


\subsection{Analysis}
% XXX Prove that the design matches the desired properties
Scope pictures should show: 
-input voltage VS output current (show with half sine wave input voltage against triangular current waveform)
-input voltage VS input current, (ie power factor)

\section{Results}
Show measured efficiency of power converter
Show dimmability


\subsection{Simulation Results}
Show graphs that agree with scope pics

\subsection{Experimental Results}


\subsection{Discussion}
The current controlled power converter circuit presented in this paper is also suitable for uses other than LEDs, such as for for charging batteries. A transistor and 3 resistors can be added to the circuit in order to provide voltage regulation in parallel with the current regulator. If both are made adjustable, it is a very nice current/voltage adjustable power supply.
what more is there to say?

















% An example of a floating figure using the graphicx package.
% Note that \label must occur AFTER (or within) \caption.
% For figures, \caption should occur after the \includegraphics.
% Note that IEEEtran v1.7 and later has special internal code that
% is designed to preserve the operation of \label within \caption
% even when the captionsoff option is in effect. However, because
% of issues like this, it may be the safest practice to put all your
% \label just after \caption rather than within \caption{}.
%
% Reminder: the "draftcls" or "draftclsnofoot", not "draft", class
% option should be used if it is desired that the figures are to be
% displayed while in draft mode.
%
%\begin{figure}[!t]
%\centering
%\includegraphics[width=2.5in]{myfigure}
% where an .eps filename suffix will be assumed under latex, 
% and a .pdf suffix will be assumed for pdflatex; or what has been declared
% via \DeclareGraphicsExtensions.
%\caption{Simulation Results}
%\label{fig_sim}
%\end{figure}

% Note that IEEE typically puts floats only at the top, even when this
% results in a large percentage of a column being occupied by floats.


% An example of a double column floating figure using two subfigures.
% (The subfig.sty package must be loaded for this to work.)
% The subfigure \label commands are set within each subfloat command, the
% \label for the overall figure must come after \caption.
% \hfil must be used as a separator to get equal spacing.
% The subfigure.sty package works much the same way, except \subfigure is
% used instead of \subfloat.
%
%\begin{figure*}[!t]
%\centerline{\subfloat[Case I]\includegraphics[width=2.5in]{subfigcase1}%
%\label{fig_first_case}}
%\hfil
%\subfloat[Case II]{\includegraphics[width=2.5in]{subfigcase2}%
%\label{fig_second_case}}}
%\caption{Simulation results}
%\label{fig_sim}
%\end{figure*}
%
% Note that often IEEE papers with subfigures do not employ subfigure
% captions (using the optional argument to \subfloat), but instead will
% reference/describe all of them (a), (b), etc., within the main caption.


% An example of a floating table. Note that, for IEEE style tables, the 
% \caption command should come BEFORE the table. Table text will default to
% \footnotesize as IEEE normally uses this smaller font for tables.
% The \label must come after \caption as always.
%
%\begin{table}[!t]
%% increase table row spacing, adjust to taste
%\renewcommand{\arraystretch}{1.3}
% if using array.sty, it might be a good idea to tweak the value of
% \extrarowheight as needed to properly center the text within the cells
%\caption{An Example of a Table}
%\label{table_example}
%\centering
%% Some packages, such as MDW tools, offer better commands for making tables
%% than the plain LaTeX2e tabular which is used here.
%\begin{tabular}{|c||c|}
%\hline
%One & Two\\
%\hline
%Three & Four\\
%\hline
%\end{tabular}
%\end{table}


% Note that IEEE does not put floats in the very first column - or typically
% anywhere on the first page for that matter. Also, in-text middle ("here")
% positioning is not used. Most IEEE journals/conferences use top floats
% exclusively. Note that, LaTeX2e, unlike IEEE journals/conferences, places
% footnotes above bottom floats. This can be corrected via the \fnbelowfloat
% command of the stfloats package.



\section{Conclusion}
The conclusion goes here.




% conference papers do not normally have an appendix


% use section* for acknowledgement
\section*{Acknowledgment}


The authors would like to thank...





% trigger a \newpage just before the given reference
% number - used to balance the columns on the last page
% adjust value as needed - may need to be readjusted if
% the document is modified later
%\IEEEtriggeratref{8}
% The "triggered" command can be changed if desired:
%\IEEEtriggercmd{\enlargethispage{-5in}}

% references section

% can use a bibliography generated by BibTeX as a .bbl file
% BibTeX documentation can be easily obtained at:
% http://www.ctan.org/tex-archive/biblio/bibtex/contrib/doc/
% The IEEEtran BibTeX style support page is at:
% http://www.michaelshell.org/tex/ieeetran/bibtex/
\bibliographystyle{IEEEtran}
% argument is your BibTeX string definitions and bibliography database(s)
%\bibliography{IEEEabrv,../bib/paper}
\bibliography{ref}
%
% <OR> manually copy in the resultant .bbl file
% set second argument of \begin to the number of references
% (used to reserve space for the reference number labels box)
%\begin{thebibliography}{1}

%\bibitem{IEEEhowto:kopka}
%H.~Kopka and P.~W. Daly, \emph{A Guide to \LaTeX}, 3rd~ed.\hskip 1em plus
%  0.5em minus 0.4em\relax Harlow, England: Addison-Wesley, 1999.

%\end{thebibliography}




% that's all folks
\end{document}


